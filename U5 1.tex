\documentclass{article}

\usepackage{amsfonts}
\usepackage{amsthm}
\usepackage{amsmath}
\usepackage[utf8]{inputenc}
\usepackage[spanish]{babel}

\newtheorem{theorem}{Teorema}[section]
\newtheorem*{theorem*}{Teorema}
\newtheorem{prop}[theorem]{Proposición}
\newtheorem*{prop*}{Proposición}
\newtheorem{cor}[theorem]{Corolario}
\newtheorem*{cor*}{Corolario}
\newtheorem{lemma}[theorem]{Lema}
\newtheorem*{lemma*}{Lema}

\theoremstyle{definition}
\newtheorem{exmp}[theorem]{Ejemplo}
\newtheorem{definition}[theorem]{Definición}

\newcommand{\N}{\mathbb{N}}
\newcommand{\Z}{\mathbb{Z}}
\newcommand{\R}{\mathbb{R}}
\newcommand{\Q}{\mathbb{Q}}
\newcommand{\C}{\mathbb{C}}
\newcommand{\I}{\mathbb{I}}
\newcommand{\F}{\mathbb{F}}
\newcommand{\A}{\mathcal{A}}
\newcommand{\B}{\mathcal{B}}
\newcommand{\U}{\mathcal{U}}
\newcommand{\cc}{\mathfrak{c}}
\newcommand{\norm}[1]{\left\lVert#1\right\rVert}
\newcommand{\seq}[1]{\left \{ #1_n \right \}_{n \in \N}}
\newcommand{\seqq}[2]{\left \{ (#1_n,#2_n) \right \}_{n \in \N}}


\DeclareMathOperator{\cl}{cl}
\DeclareMathOperator{\interior}{int}
\DeclareMathOperator{\dom}{dom}

\renewcommand{\baselinestretch}{1.5}

\begin{document}


\section*{Unidad 5: vectores y valores propios}

\begin{enumerate}
	\item Dada una matriz $A \in \R^{n \times n}$, demuestre que si $X$ es una matriz cuadrada no singular entonces $A$ y $X^{-1} A X$ tienen el mismo polinomio característico.
	
	\begin{proof}
		Usaremos que la función  $B \mapsto \det(B)$ es multiplicativa. Sean $p_1,p_2$ los polinomios característicos de $A$ y $X^{-1}AX$, respectivamente. Entonces
		\begin{align}
			p_2 (t) & = \det ( X^{-1}AX - tI) \\
			& = \det( X^{-1} ( AX - t X)) \\
			& = \det( X^{-1} ( A - tI) X) \\
			& = \det (X^{-1}) \det (A -tI) \det(X) \\
			& = \frac{1}{\det (X)} \det (A-tI) \det( X) \\
			& = \det (A-tI) \\
			&  = p_1 (t).
		\end{align}
		
	Por lo tanto, $A$ y $X^{-1}AX$ tienen el mismo polinomio característico.
	\end{proof}
	
\item  Dada una matriz $A \in \R^{n \times n}$ y sus respectivos valores propios $\lambda_1, \dots ,\lambda_n$, demuestre que
	\begin{align}
		\det (A ) = \prod _{j=1}^n \lambda_j.
	\end{align}
	
	\begin{proof}
		Recordemos que $A$ es diagonalizable sobre el campo de los complejos, por lo que existe una matriz invertible $N$ con coeficientes complejos tal que
			\begin{align}
				N^{-1}AN = D,
			\end{align}
		donde
			\begin{align}
				D  = diag(\lambda_1, \dots, \lambda_n).
			\end{align}
			
		Usaremos que la función $B \mapsto \det (B)$ es multiplicativa:
			\begin{align}
				\det (D) & = \det ( N^{-1}) \det (A) \det (N) \\
				& = \frac{1}{\det(N)} \det (A) \det (N) \\
				& = \det (A).
			\end{align}
		Por lo tanto, 
			\begin{align}
				\det (A) & =   \det(D) \\
				& = \prod_{j=1}^n \lambda_j,
			\end{align}
		donde la última igualdad se da porque el determinante de una matriz diagonal es igual al producto de las entradas de su diagonal.
	\end{proof}
	
\item 
\item 

\item Muestre que los valores propios de una matriz simétrica con coeficientes reales de $ 2 \times 2$ son reales. ¿Cómo quedaría el resultado en general para una matriz simétrica con coeficientes en los reales de $n \times n$?

\begin{proof}
	scá
	
\end{proof}

\item Calcule la descomposición de Schur de la matrix
	\begin{align}
		A = \begin{pmatrix}
			1 & 2 \\
			2 & 3
		\end{pmatrix}
	\end{align}

\textbf{Solución:} Primero calculamos el polinomio característico de $A$:
	\begin{align}
		p(t) & = t^2-tr A t+\det A \\
		&  = t^2 -4t-1 \\
		& = \left (t- \left(2+\sqrt{5} \right) \right) \left( t-\left(2-\sqrt{5}\right) \right)
	\end{align}
	
	Para encontrar un valor propio asociado a $\lambda_1 = 2+\sqrt{5}$, resolvemos el sistema
		\begin{align}
			0 & = (A-\lambda_1 I) v \\
			& = \begin{pmatrix}
				-1 - \sqrt{5} & 2 \\
				2 & 1-\sqrt{5}
			\end{pmatrix} \begin{pmatrix}
			x \\ y
			\end{pmatrix} \\
			& = \begin{pmatrix}
				-(1+\sqrt{5}) x +2y \\
				2x +(1-\sqrt{5})y
			\end{pmatrix}.
		\end{align}
	Las dos ecuaciones de este sistema son equivalentes, por lo que nos podemos deshacer de la segunda y obtenemos
		\begin{align}
			y = \frac{1+\sqrt{5}}{2} x.
		\end{align}
	Haciendo $x=2$, obtenemos el vector propio
		\begin{align}
			v = \begin{pmatrix}
				2 \\ 1+\sqrt{5}
			\end{pmatrix}
		\end{align}
	Para obtener un vector propio asociado a $\lambda_2 = 2-\sqrt{5}$ basta conjugar $v$ respecto a $\sqrt{5}$, obteniendo 
		\begin{align}
			w = \begin{pmatrix}
			2 \\
			1-\sqrt{5}
			\end{pmatrix}
		\end{align}
		
	Notemos que (afortunadamente), $v$ y $w$ ya son ortogonales, por lo que para obtener una base ortonormal para $\R^2$ compuesta por vectores propios de $A$ basta normalizar a $v$ y $w$. Definimos
		\begin{align}
			u_1 & = \frac{u}{\norm{u}} = \frac{1}{\sqrt{10+2\sqrt{5}}} \begin{pmatrix}
			2 \\ 1+\sqrt{5}
			\end{pmatrix} \\
			 u_2 & = \frac{w}{\norm{w}} = \frac{1}{\sqrt{ 10-2\sqrt{5} } } \begin{pmatrix} 2 \\ 1- {\sqrt{5}}
			 \end{pmatrix} 
		\end{align}
	De esta forma, si consideramos
		\begin{align}
			Q = \begin{pmatrix}
				u_1 & u_2
			\end{pmatrix}
			= \begin{pmatrix}
			\frac{2}{\sqrt{10+2\sqrt{5}}} &  \frac{2}{\sqrt{ 10-2\sqrt{5} }} \\
			\frac{1+\sqrt{5}}{\sqrt{10+2\sqrt{5}}} & \frac{1- \sqrt{5}}{\sqrt{ 10-2\sqrt{5} }}
			\end{pmatrix}
		\end{align}
	y 
		\begin{align}
			D = \begin{pmatrix}
					2+ \sqrt{5} & 0 \\
					0 & 2- \sqrt{5}
			\end{pmatrix}
		\end{align}
	entonces
		\begin{align}
			A = Q D Q^T,
		\end{align}
	esta es la factorización de Schur de $A$.
\item
	

	
\end{enumerate} 

\end{document}
